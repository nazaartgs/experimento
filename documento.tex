% --- INICIO DE PLANTILLA PERSONALIZADA ---
\documentclass[oneside,11pt]{Latex/Classes/PhDthesisPSnPDF}

% --- Paquetes Originales ---
\usepackage{blindtext}
\usepackage{actuarialangle} 
\usepackage{tabularx}
\usepackage{float}
\usepackage{multirow, array}
\usepackage[usenames,dvipsnames,svgnames,table]{xcolor}
\usepackage{longtable}
\usepackage{latexsym,amsmath,amssymb,amsfonts}
\usepackage{enumitem}
\usepackage{xparse}
\usepackage{booktabs}
\usepackage[round, sort, numbers]{natbib}  
\usepackage{adjustbox}
\usepackage[utf8]{inputenc}
\usepackage{caption}
\usepackage{graphicx}
\usepackage{hyperref}
\usepackage{cleveref}

% Definición de colores
\definecolor{miblue}{RGB}{68, 114, 196}

% --- CAMBIO CRÍTICO 1: Usar \input para macros en el preámbulo ---
% \include da problemas antes del begin document. \input es seguro.
\input{Latex/Macros/MacroFile1}

% --- Definiciones de seguridad (por si la clase falla al cargar alguna variable) ---
\providecommand{\facultad}[1]{\def\lafacultad{#1}}
\providecommand{\escudofacultad}[1]{\def\elescudo{#1}}
\providecommand{\degree}[1]{\def\elgrado{#1}}
\providecommand{\director}[1]{\def\eldirector{#1}}
\providecommand{\lugar}[1]{\def\ellugar{#1}}
\providecommand{\degreedate}[1]{\def\lafecha{#1}}

% --- Configuración para bloques de código de R (Pandoc) ---

%%%%%%%%%%%%%%%%%%%%%%%%%%%%%%%%%%%%%%%%%%%%%%%%%%%%%%%%%%%%%%%%%%%%%%%%%%%%%%%%
%                         DATOS (Conectados a RMarkdown)                       %
%%%%%%%%%%%%%%%%%%%%%%%%%%%%%%%%%%%%%%%%%%%%%%%%%%%%%%%%%%%%%%%%%%%%%%%%%%%%%%%%
\title{Analisis del Mercado de Videojuegos}
\author{true \\ true}

% Lógica para Facultad: Si la defines en el YAML la usa, si no, usa la default

\facultad{Facultad de Ciencias Económicas y Sociales \\ Escuela de
Estadistica y Ciencias Actuariales}
  \escudofacultad{Latex/Classes/Escudos/faces}

\degree{Computación I}
\director{Prof.~Jesus Ochoa \\ Prof.~Oliver Triveño}
\degreedate{Febrero 2026} 
\lugar{Caracas}

\portadatrue 

% Metadatos PDF
\keywords{}
\subject{}

% --- CORRECCIÓN PARA PANDOC NUEVO (Imágenes) ---
\makeatletter
\providecommand{\pandocbounded}[1]{#1}
\makeatother


%%%%%%%%%%%%%%%%%%%%%%%%%%%%%%%%%%%%%%%%%%%%%%%%%%%%%
%                   DOCUMENTO                       %
%%%%%%%%%%%%%%%%%%%%%%%%%%%%%%%%%%%%%%%%%%%%%%%%%%%%%
\begin{document}

\maketitle

%%%%%%%%%%%%%%%%%%%%%%%%%%%%%%%%%%%%%%%%%%%%%%%%%%%%%
%                  PRÓLOGO                          %
%%%%%%%%%%%%%%%%%%%%%%%%%%%%%%%%%%%%%%%%%%%%%%%%%%%%%
\frontmatter

% Puedes agregar dedicatorias desde el YAML si quieres, o dejarlas fijas aquí

%%%%%%%%%%%%%%%%%%%%%%%%%%%%%%%%%%%%%%%%%%%%%%%%%%%%%
%                   ÍNDICES                         %
%%%%%%%%%%%%%%%%%%%%%%%%%%%%%%%%%%%%%%%%%%%%%%%%%%%%%
\setcounter{secnumdepth}{3}
\setcounter{tocdepth}{3}

\tableofcontents



\cleardoublepage

  \cleardoublepage       % Empieza en página derecha (estándar en tesis)
  \chapter*{Resumen}     % Crea el título "Resumen" sin numeración (Capítulo *)
  \addcontentsline{toc}{chapter}{Resumen} % (Opcional) Lo añade al índice
  
  En el presente informe se realiza un análisis exploratorio de los
datos de ventas de videojuegos a nivel global. Se examinan las
tendencias históricas, la predominancia de géneros y el desempeño por
regiones (NA, EU, JP), aplicando técnicas estadísticas descriptivas y
visualización de datos con
R.             % Aquí se pega el texto que escribiste en el YAML
  
  \cleardoublepage

%%%%%%%%%%%%%%%%%%%%%%%%%%%%%%%%%%%%%%%%%%%%%%%%%%%%%
%                   CONTENIDO (El Corazón)          %
%%%%%%%%%%%%%%%%%%%%%%%%%%%%%%%%%%%%%%%%%%%%%%%%%%%%%
\mainmatter
\def\baselinestretch{1.5}

% --- CAMBIO CRÍTICO 2: Aquí RMarkdown inyecta todo tu texto ---
\providecommand{\tightlist}{%
  \setlength{\itemsep}{0pt}\setlength{\parskip}{0pt}}

\newpage

\section{1. Introducción}\label{introducciuxf3n}

\section{2. Planteamiento del
Problema.}\label{planteamiento-del-problema.}

En la última década, el mercado laboral global ha experimentado una
transformación sin precedentes. Factores como la digitalización
acelerada, la normalización del trabajo remoto y la demanda de
habilidades técnicas específicas han redefinido cómo se valoran los
puestos de trabajo. Ya no basta con mirar el ``título del cargo''; la
compensación económica ahora parece estar sujeta a una red compleja de
variables interconectadas.

¿Es el tamaño de la empresa un predictor fiable del sueldo? ¿Existe una
penalización salarial por elegir la modalidad remota, o es ahora un
estándar competitivo? ¿Qué tanto peso real tienen ciertas habilidades
(skills) frente a la ubicación geográfica?. A pesar de la abundancia de
ofertas de trabajo, existe una falta de claridad sobre cuáles son los
verdaderos determinantes del salario en el mercado actual, por lo cual,
este estudio busca aclarar esas dudas mediante la estadística
descriptiva y el uso de la plataforma GitHub y lenguajes de programación
conocidos como R y Python.

\section{3. Objetivos}\label{objetivos}

\subsection{3.1 Objetivo General}\label{objetivo-general}

Analizar el panorama laboral de los sectores tecnológicos emergentes
mediante estadística descriptiva para identificar las variables que más
impactan en la oferta salarial y la demanda de talento.

\subsection{3.2 Objetivos Específicos}\label{objetivos-especuxedficos}

\begin{itemize}
\item
  Determinar el promedio, mediana y desviación estándar de los salarios
  por industria para identificar cuál es la más rentable.
\item
  Comparar las medias salariales entre puestos remotos y presenciales
  para verificar si existe una ``penalización'' o beneficio económico
  por la modalidad.
\item
  Identificar las ciudades con mayores ofertas de empleo y su nivel de
  remuneración promedio.
\item
  Clasificar las habilidades más solicitadas en cada sector y su
  frecuencia de aparición en las vacantes.
\item
  Evaluar si el tamaño de la empresa (Small, Medium, Large) influye
  significativamente en los rangos salariales ofrecidos.
\end{itemize}

\section{4. Marco de Referencia}\label{marco-de-referencia}

\subsection{4.1 Marco Teórico}\label{marco-teuxf3rico}

\begin{itemize}
\tightlist
\item
  Teoría del Capital Humano: Propone que el salario está directamente
  relacionado con la productividad del trabajador, la cual se mide a
  través de sus habilidades y educación.
\item
  Teoría de los Salarios Compensatorios: Sugiere que las características
  no monetarias de un trabajo (como la modalidad remota o el tamaño de
  la empresa) afectan el sueldo.
\item
  Segmentación del Mercado Laboral: Plantea que el mercado no es
  uniforme. Variables como industry y company\_size crean ``segmentos''
  donde las reglas de compensación cambian drásticamente.
\end{itemize}

\subsection{4.2 Marco Metodológico}\label{marco-metodoluxf3gico}

\section{5. Enfoque y Tipo de
Investigación}\label{enfoque-y-tipo-de-investigaciuxf3n}

Tipo descriptivo, correlacional no experimental y de enfoque
cuantitativo.

\section{6. Fuente de Datos}\label{fuente-de-datos}

Los datos a evaluar se encuentran alojados en el repositorio
nazaartgs/Jobs\_and\_skills: Un estudio estadístico creado con R y
python para analizar las habilidades requeridas en el mercado laboral de
la tecnología el cual ya se encuentra depurado y listo para su revisión.

\section{7. Período de Referencia}\label{peruxedodo-de-referencia}

\section{8. Procesamiento y Análisis de
Datos}\label{procesamiento-y-anuxe1lisis-de-datos}

El análisis se realiza con \textbf{R} y las librerías \texttt{dplyr} y
\texttt{ggplot2}. El proceso consta de: 1. \textbf{Limpieza de Datos:}
Carga del dataset, manejo de valores atípicos o no válidos (ej. precios
en cero, número de habitaciones irreal), y corrección de tipos de datos.
2. \textbf{Análisis Descriptivo:} Cálculo de estadísticas resumen. 3.
\textbf{Visualización:} Creación de gráficos para ilustrar
distribuciones y relaciones. Se utiliza una paleta de colores
consistente inspirada en ``La noche estrellada''. 4.
\textbf{Tabulación:} Generación de tablas formateadas con \texttt{kable}
y \texttt{kableExtra} para una presentación clara de los resultados.

\#9. Resultados y Análisis

\section{10. Análisis}\label{anuxe1lisis}

%%%%%%%%%%%%%%%%%%%%%%%%%%%%%%%%%%%%%%%%%%%%%%%%%%%%%
%                   REFERENCIAS                     %
%%%%%%%%%%%%%%%%%%%%%%%%%%%%%%%%%%%%%%%%%%%%%%%%%%%%%
% Mantenemos la estructura natbib de tu plantilla
\renewcommand{\bibname}{Lista de Referencias}
\bibliographystyle{apalike} 
\bibliography{bibliografia.bib} 

%%%%%%%%%%%%%%%%%%%%%%%%%%%%%%%%%%%%%%%%%%%%%%%%%%%%%
%                   APÉNDICES                       %
%%%%%%%%%%%%%%%%%%%%%%%%%%%%%%%%%%%%%%%%%%%%%%%%%%%%%

\end{document}
